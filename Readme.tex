\documentclass[11pt]{article}

\linespread{1.3}
\usepackage[top=1in, bottom=1.25in, left=1.25in, right=1.25in]{geometry}
\usepackage{amsmath}
\usepackage{graphicx}
\numberwithin{equation}{section}
\usepackage{xcolor}
\usepackage{amsfonts}

\begin{document}
\title{Augmented Plane Wave Readme}
\author{Jacob Mankenberg}
\maketitle



This should be fun. I hope it goes well. Just following the structure and algorithm as discussed in \textbf{Augmented Plane Wave Method} by Terry Loucks from \textit{1967}. Is it outdated? Perhaps. But that doesn't really matter since the point is to hash out the details and implement it efficiently and accurately. I want to eventually compare it to results from different band structure methods (Tight-Binding, Hartree-Fock, DFT and the likes). This should be the easiest one so hopefully I can build it from the ground up by myself.

\section{Theoretical Discussion of the Method}
I'll begin by restating the Bloch Theorem:
\begin{equation}
\psi_{\vec{k}}(\vec{r} + \vec{R}) = e^{-i \vec{k}\cdot\vec{R}} \psi_{\vec{k}}(\vec{r})
\end{equation}
It is the statement that the orbital wavefunctions of bound electrons in a crystal are periodic but for a phase factor. $\vec{R}$ is any lattice vector and the wavevector $\vec{k}$ is of course a new quantum number analogous to momentum but for the case of periodic translational symmetry - discrete symmetry. We tend to call it the crystal momentum but I think it's more intuitive to consider it as a mathematical product of the expansion of our wavefunction in a Fourier series with the periodicity of the lattice. By this I mean that the wavefunction can be an expansion of its spatial harmonics.

Then it's straightforward to see that if the wavevector $\vec{k}$ is a reciprocal lattice vector,
\begin{equation}
e^{-i\vec{k}\cdot\vec{R}} = e^{-i2\pi\cdot integer} = 1
\end{equation}
This statement is not a restriction but a freedom. It means that $\vec{k}$ does not uniquely label the state and is invariant under the addition of a reciprocal lattice vector. We can thus work wholly in the 1st Brillouin zone, the volume in reciprocal space enclosed by Bragg planes.

\paragraph{Muffin-Tin Potential}
The muffin-tin approximation builds off the cellular method where we simply solve the Schrodinger equation in the Brillouin zone. The cellular method was first proposed by Wigner and Seitz and used to calculate the 2s band for sodium (I think). A difficulty with the cellular method is that the continuity of the wavefunction and its derivative must be satisfied over the boundary of the primitive cell, which could be, in principle, any arbitrary shape. Another is that the single ionic potential over the cell isn't the best approximation, in addition to not being smooth over the boundary. The muffin-tin potential takes care of both of these problems by approximating the cell, or more accurately the potential inside it, as a sphere
\begin{align*}
U(\vec{r}) = &V(|\vec{r} - \vec{R}|), |\vec{r} - \vec{R}| < r_0 \\
= &V(r_0) = 0, |\vec{r} - \vec{R}| > r_0
\end{align*}
It's clear that both aforementioned problems are resolved by this approximation. It turns out it's actually a pretty good one too.

\paragraph{APW Function}
The APW method is built around an expansion of the wavefunction in \textit{augmented plane waves}, or just fancy plane waves. These APW functios $\chi_{\vec{k}}$ are defined as such:
\begin{enumerate}
\item In the interstitial region - between cores - we simply have a plane wave $\chi_{\vec{k}} = e^{i\vec{k}\cdot\vec{r}}$. This makes sense to begin with, since in this space we are effectively dealing with a free particle. It's important to note however, that there is no relation between the energy and the wavevector a priori. Therefore, a single $\chi$ does not satisfy the Schrodinger equation but rather a superposition of $\chi_{\vec{k}}$'s.

\item $\chi$ is continuous at the boundary although not necessarily smooth

\item Inside the core region, $\chi$ \textit{does} satisfy the Schrodinger equation and is taken as a superposition of spherical solutions $\chi_{\vec{k}} = \sum_{l,m}A_{lm}Y_{lm}(\hat{n})R_l(r)$. Since there is no wavevector in this equation, $H\chi_{\vec{k}} = E\chi_{\vec{k}}$ where $E$ is independent of $\vec{k}$, the $\vec{k}$ value is acquired through the boundary condition.
\end{enumerate}

To show this last point we can see how the continuity is enforced by specifying a specific set of coefficients $A_{lm}$. Just outside any sphere, $\chi_{\vec{k}} = e^{i\vec{k}\cdot\vec{r}} = d^{i\vec{k}\cdot(\vec{r} + \vec{\rho}} = e^{i\vec{k}\cdot\vec{r}}e^{i\vec{k}\cdot\vec{\rho}}$. Expanding the last plane wave in spherical harmonics,
\begin{equation}
e^{i\vec{k}\cdot\vec{\rho}} = 4\pi\sum_{l,m}i^l j_l(k\rho)Y^*_{lm}(\hat{k})Y_{lm}(\hat{n})
\end{equation}
so that
\begin{equation}
e^{i\vec{k}\cdot\vec{r}} = e^{i\vec{k}\cdot\vec{r}_{\nu}}4\pi\sum_{l,m}i^l j_l(k\rho)Y^*_{lm}(\hat{k})Y_{lm}(\hat{n})
\end{equation}
We impose continuity by asserting that $\chi_{\vec{k}}$ inside and outside the sphere are equal at the interface $\rho = S_{\nu}$. From the orthogonality of spherical harmonics,
\begin{equation}
A_{lm} = 4\pi i^l j_l(kS_{\nu})Y^*_{lm}(\hat{k})/R_l(S_{\nu})
\end{equation}
and we see that in fact, our wavevector shows up in the plane wave and in the argument of a spherical harmonic. Thus as promised, the $\vec{k}$ value of $\chi$ inside the sphere is acquired through the boundary condition and shows up in the coefficient.

To summarize,
\begin{align}
\chi_{\vec{k}}(\vec{r}) = &e^{i\vec{k}\cdot\vec{r}} \\
= &4\pi e^{i\vec{k}\cdot\vec{r}_{\nu}}\sum_{l,m}i^l j_l(kS_{\nu})Y^*_{lm}(\hat{k})Y_{lm}(\hat{n})R_l(r)/R_l(S_{\nu})
\end{align}
and is continuous but not smooth across the muffin-tin boundary. 

\paragraph{Variational Expression for the Energy}
The APW method is a way to approximate the crystal Schrodinger equation wavefunction by a superposition of APW functions of the same energy. We'll write this as
\begin{equation}
\phi_{\vec{k}}(\vec{r}) = \sum_{\vec{K}}c_{\vec{K}}\chi_{\vec{k} + \vec{K}}(\vec{r})
\end{equation}
since the APW satisfies the Bloch condition with wavevector $\vec{k}$ for any reciprocal lattice vector $\vec{K}$. Since we did not impose smoothness on the APW's, the derivative is discontinuous across the boundary and therefore the Laplacian will have delta function singularities. To get around this we work with a variational principle.

What we mean explicitly, is that if $\phi_t$ and $E_t$ are the true solutions to the Schrodinger equation
\begin{equation}
H\phi_t = E_t\phi_t
\end{equation}
with Hamiltonian $H = -\nabla^2 + V(\vec{r})$, then first order variations $E = E_t + \delta E$, $\phi_I = \phi_t + \delta \phi_I$ and $\phi_{II} = \phi_t + \delta\phi_{II}$ will yield a variational equation
\begin{align} 
E\int\limits_{I+II}\phi^*\phi dV = &\int\limits_{I+II}\phi^*H\phi dV \nonumber \\
+ \frac{1}{2}&\int\limits_{Surf}\big[(\phi_{II}+\phi_I)(\partial_{\rho}\phi^*_{II} + \partial_{\rho}\phi^*_I) - (\phi_{II} + \phi_I)^*(\partial_{\rho}\phi_{II} - \partial_{\rho}\phi_I)\big]dS
\end{align}
if and only if $\delta E = 0$. This will eventually generate the energy profile $E(\vec{k})$ that we seek: the band structure. At this point it would be customary to give a proof or a derivation...if this were a physics text. It is not. Go look it up yourself if you don't believe me.

Since we derived this expression by considering the general case of noncontinuous $\phi$ across the boundaries, the variational equation is slightly more complicated than need be. Our proceeding discussion imposed continuity on those function and thus the variation simplifies to
\begin{equation} \label{eq:varPrinc}
E\int\limits_{I+II}\phi^*\phi dV = \int\limits_{I+II}\phi^*H\phi dV = -\frac{1}{2}\int\limits_{\mathrm{Surf}}\big[(\phi_{II} + \phi_I)^*(\partial_{\rho}\phi_{II} - \partial_{\rho}\phi_I)\big]dS
\end{equation}
This is the variational equation we'll be using throughout the rest of the discussion.

\paragraph{APW Matrix Elements}

The matrix elements are the computational implementations of the APW method. We want to use the variational equation \ref{eq:varPrinc} to find the optimal expension parameters, or rather the stationary parameters in the expansion of the wavefunction:
\begin{equation}
\frac{\partial E}{\partial c_i} = 0, \hspace{1cm} i = 1,2,...
\end{equation}


\section{Practical Aspects and the Algorithm}

\section{Code and ``Documentation"}

\paragraph{Spherical Harmonic Generation}
The code for generating the numerical results of the spherical harmonics is inspired from the paper \cite{spherHarm}. It's not really important but perhaps a good idea to keep in mind that the spherical harmonics are real and $4\pi$ normalized. Real spherical harmonics are defined with respect to "normal" spherical harmonics as \textcolor{red}{equation for multiple m}.

\paragraph{Spherical Bessel Function Generation}
I'll try 2 different algorithms just for fun. One will be the method proposed by Harrison in \cite{spherBessel} which is actually pretty interesting. The other is done in a similar fashion to the way the spherical harmonics are generated. It'll also be fun. Let's see which one is faster.



\begin{thebibliography}{99}

\bibitem{loucks}
  T.L. Loucks, \textit{Augmented Plane Wave Method}, W. A. Benjamin, Menlo Park, California
  
\bibitem{am}
  N. W. Ashcroft, N. D. Mermin, \textit{Solid State Physics}, Cengage, Stanford, California
  
\bibitem{spherHarm}
  T. Limpanuparb and J. Milthorpe. arXiv: 1410.1748 [physics.chem-ph], (2014)
  
\bibitem{spherBessel}
  J. Harrison. \textit{Fast and Accurate Bessel Function Computation} Intel Corporation \\
  \textit{Retrieved at } https://www.cl.cam.ac.uk/~jrh13/papers/bessel.pdf
  

\end{thebibliography}


\end{document}